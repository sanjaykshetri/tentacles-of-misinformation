% Options for packages loaded elsewhere
% Options for packages loaded elsewhere
\PassOptionsToPackage{unicode}{hyperref}
\PassOptionsToPackage{hyphens}{url}
%
\documentclass[
  letterpaper,
]{book}
\usepackage{xcolor}
\usepackage[margin=1in]{geometry}
\usepackage{amsmath,amssymb}
\setcounter{secnumdepth}{5}
\usepackage{iftex}
\ifPDFTeX
  \usepackage[T1]{fontenc}
  \usepackage[utf8]{inputenc}
  \usepackage{textcomp} % provide euro and other symbols
\else % if luatex or xetex
  \usepackage{unicode-math} % this also loads fontspec
  \defaultfontfeatures{Scale=MatchLowercase}
  \defaultfontfeatures[\rmfamily]{Ligatures=TeX,Scale=1}
\fi
\usepackage{lmodern}
\ifPDFTeX\else
  % xetex/luatex font selection
\fi
% Use upquote if available, for straight quotes in verbatim environments
\IfFileExists{upquote.sty}{\usepackage{upquote}}{}
\IfFileExists{microtype.sty}{% use microtype if available
  \usepackage[]{microtype}
  \UseMicrotypeSet[protrusion]{basicmath} % disable protrusion for tt fonts
}{}
\makeatletter
\@ifundefined{KOMAClassName}{% if non-KOMA class
  \IfFileExists{parskip.sty}{%
    \usepackage{parskip}
  }{% else
    \setlength{\parindent}{0pt}
    \setlength{\parskip}{6pt plus 2pt minus 1pt}}
}{% if KOMA class
  \KOMAoptions{parskip=half}}
\makeatother
% Make \paragraph and \subparagraph free-standing
\makeatletter
\ifx\paragraph\undefined\else
  \let\oldparagraph\paragraph
  \renewcommand{\paragraph}{
    \@ifstar
      \xxxParagraphStar
      \xxxParagraphNoStar
  }
  \newcommand{\xxxParagraphStar}[1]{\oldparagraph*{#1}\mbox{}}
  \newcommand{\xxxParagraphNoStar}[1]{\oldparagraph{#1}\mbox{}}
\fi
\ifx\subparagraph\undefined\else
  \let\oldsubparagraph\subparagraph
  \renewcommand{\subparagraph}{
    \@ifstar
      \xxxSubParagraphStar
      \xxxSubParagraphNoStar
  }
  \newcommand{\xxxSubParagraphStar}[1]{\oldsubparagraph*{#1}\mbox{}}
  \newcommand{\xxxSubParagraphNoStar}[1]{\oldsubparagraph{#1}\mbox{}}
\fi
\makeatother

\usepackage{color}
\usepackage{fancyvrb}
\newcommand{\VerbBar}{|}
\newcommand{\VERB}{\Verb[commandchars=\\\{\}]}
\DefineVerbatimEnvironment{Highlighting}{Verbatim}{commandchars=\\\{\}}
% Add ',fontsize=\small' for more characters per line
\usepackage{framed}
\definecolor{shadecolor}{RGB}{241,243,245}
\newenvironment{Shaded}{\begin{snugshade}}{\end{snugshade}}
\newcommand{\AlertTok}[1]{\textcolor[rgb]{0.68,0.00,0.00}{#1}}
\newcommand{\AnnotationTok}[1]{\textcolor[rgb]{0.37,0.37,0.37}{#1}}
\newcommand{\AttributeTok}[1]{\textcolor[rgb]{0.40,0.45,0.13}{#1}}
\newcommand{\BaseNTok}[1]{\textcolor[rgb]{0.68,0.00,0.00}{#1}}
\newcommand{\BuiltInTok}[1]{\textcolor[rgb]{0.00,0.23,0.31}{#1}}
\newcommand{\CharTok}[1]{\textcolor[rgb]{0.13,0.47,0.30}{#1}}
\newcommand{\CommentTok}[1]{\textcolor[rgb]{0.37,0.37,0.37}{#1}}
\newcommand{\CommentVarTok}[1]{\textcolor[rgb]{0.37,0.37,0.37}{\textit{#1}}}
\newcommand{\ConstantTok}[1]{\textcolor[rgb]{0.56,0.35,0.01}{#1}}
\newcommand{\ControlFlowTok}[1]{\textcolor[rgb]{0.00,0.23,0.31}{\textbf{#1}}}
\newcommand{\DataTypeTok}[1]{\textcolor[rgb]{0.68,0.00,0.00}{#1}}
\newcommand{\DecValTok}[1]{\textcolor[rgb]{0.68,0.00,0.00}{#1}}
\newcommand{\DocumentationTok}[1]{\textcolor[rgb]{0.37,0.37,0.37}{\textit{#1}}}
\newcommand{\ErrorTok}[1]{\textcolor[rgb]{0.68,0.00,0.00}{#1}}
\newcommand{\ExtensionTok}[1]{\textcolor[rgb]{0.00,0.23,0.31}{#1}}
\newcommand{\FloatTok}[1]{\textcolor[rgb]{0.68,0.00,0.00}{#1}}
\newcommand{\FunctionTok}[1]{\textcolor[rgb]{0.28,0.35,0.67}{#1}}
\newcommand{\ImportTok}[1]{\textcolor[rgb]{0.00,0.46,0.62}{#1}}
\newcommand{\InformationTok}[1]{\textcolor[rgb]{0.37,0.37,0.37}{#1}}
\newcommand{\KeywordTok}[1]{\textcolor[rgb]{0.00,0.23,0.31}{\textbf{#1}}}
\newcommand{\NormalTok}[1]{\textcolor[rgb]{0.00,0.23,0.31}{#1}}
\newcommand{\OperatorTok}[1]{\textcolor[rgb]{0.37,0.37,0.37}{#1}}
\newcommand{\OtherTok}[1]{\textcolor[rgb]{0.00,0.23,0.31}{#1}}
\newcommand{\PreprocessorTok}[1]{\textcolor[rgb]{0.68,0.00,0.00}{#1}}
\newcommand{\RegionMarkerTok}[1]{\textcolor[rgb]{0.00,0.23,0.31}{#1}}
\newcommand{\SpecialCharTok}[1]{\textcolor[rgb]{0.37,0.37,0.37}{#1}}
\newcommand{\SpecialStringTok}[1]{\textcolor[rgb]{0.13,0.47,0.30}{#1}}
\newcommand{\StringTok}[1]{\textcolor[rgb]{0.13,0.47,0.30}{#1}}
\newcommand{\VariableTok}[1]{\textcolor[rgb]{0.07,0.07,0.07}{#1}}
\newcommand{\VerbatimStringTok}[1]{\textcolor[rgb]{0.13,0.47,0.30}{#1}}
\newcommand{\WarningTok}[1]{\textcolor[rgb]{0.37,0.37,0.37}{\textit{#1}}}

\usepackage{longtable,booktabs,array}
\usepackage{calc} % for calculating minipage widths
% Correct order of tables after \paragraph or \subparagraph
\usepackage{etoolbox}
\makeatletter
\patchcmd\longtable{\par}{\if@noskipsec\mbox{}\fi\par}{}{}
\makeatother
% Allow footnotes in longtable head/foot
\IfFileExists{footnotehyper.sty}{\usepackage{footnotehyper}}{\usepackage{footnote}}
\makesavenoteenv{longtable}
\usepackage{graphicx}
\makeatletter
\newsavebox\pandoc@box
\newcommand*\pandocbounded[1]{% scales image to fit in text height/width
  \sbox\pandoc@box{#1}%
  \Gscale@div\@tempa{\textheight}{\dimexpr\ht\pandoc@box+\dp\pandoc@box\relax}%
  \Gscale@div\@tempb{\linewidth}{\wd\pandoc@box}%
  \ifdim\@tempb\p@<\@tempa\p@\let\@tempa\@tempb\fi% select the smaller of both
  \ifdim\@tempa\p@<\p@\scalebox{\@tempa}{\usebox\pandoc@box}%
  \else\usebox{\pandoc@box}%
  \fi%
}
% Set default figure placement to htbp
\def\fps@figure{htbp}
\makeatother





\setlength{\emergencystretch}{3em} % prevent overfull lines

\providecommand{\tightlist}{%
  \setlength{\itemsep}{0pt}\setlength{\parskip}{0pt}}



 


\makeatletter
\@ifpackageloaded{tcolorbox}{}{\usepackage[skins,breakable]{tcolorbox}}
\@ifpackageloaded{fontawesome5}{}{\usepackage{fontawesome5}}
\definecolor{quarto-callout-color}{HTML}{909090}
\definecolor{quarto-callout-note-color}{HTML}{0758E5}
\definecolor{quarto-callout-important-color}{HTML}{CC1914}
\definecolor{quarto-callout-warning-color}{HTML}{EB9113}
\definecolor{quarto-callout-tip-color}{HTML}{00A047}
\definecolor{quarto-callout-caution-color}{HTML}{FC5300}
\definecolor{quarto-callout-color-frame}{HTML}{acacac}
\definecolor{quarto-callout-note-color-frame}{HTML}{4582ec}
\definecolor{quarto-callout-important-color-frame}{HTML}{d9534f}
\definecolor{quarto-callout-warning-color-frame}{HTML}{f0ad4e}
\definecolor{quarto-callout-tip-color-frame}{HTML}{02b875}
\definecolor{quarto-callout-caution-color-frame}{HTML}{fd7e14}
\makeatother
\makeatletter
\@ifpackageloaded{bookmark}{}{\usepackage{bookmark}}
\makeatother
\makeatletter
\@ifpackageloaded{caption}{}{\usepackage{caption}}
\AtBeginDocument{%
\ifdefined\contentsname
  \renewcommand*\contentsname{Table of contents}
\else
  \newcommand\contentsname{Table of contents}
\fi
\ifdefined\listfigurename
  \renewcommand*\listfigurename{List of Figures}
\else
  \newcommand\listfigurename{List of Figures}
\fi
\ifdefined\listtablename
  \renewcommand*\listtablename{List of Tables}
\else
  \newcommand\listtablename{List of Tables}
\fi
\ifdefined\figurename
  \renewcommand*\figurename{Figure}
\else
  \newcommand\figurename{Figure}
\fi
\ifdefined\tablename
  \renewcommand*\tablename{Table}
\else
  \newcommand\tablename{Table}
\fi
}
\@ifpackageloaded{float}{}{\usepackage{float}}
\floatstyle{ruled}
\@ifundefined{c@chapter}{\newfloat{codelisting}{h}{lop}}{\newfloat{codelisting}{h}{lop}[chapter]}
\floatname{codelisting}{Listing}
\newcommand*\listoflistings{\listof{codelisting}{List of Listings}}
\makeatother
\makeatletter
\makeatother
\makeatletter
\@ifpackageloaded{caption}{}{\usepackage{caption}}
\@ifpackageloaded{subcaption}{}{\usepackage{subcaption}}
\makeatother
\usepackage{bookmark}
\IfFileExists{xurl.sty}{\usepackage{xurl}}{} % add URL line breaks if available
\urlstyle{same}
\hypersetup{
  pdftitle={The Tentacles of Misinformation},
  pdfauthor={Sanjay Kumar Chhetri},
  hidelinks,
  pdfcreator={LaTeX via pandoc}}


\title{The Tentacles of Misinformation}
\usepackage{etoolbox}
\makeatletter
\providecommand{\subtitle}[1]{% add subtitle to \maketitle
  \apptocmd{\@title}{\par {\large #1 \par}}{}{}
}
\makeatother
\subtitle{Behavioral Science, NLP, and Predictive Analytics for
Misinformation Susceptibility}
\author{Sanjay Kumar Chhetri}
\date{2026-01-16}
\begin{document}
\frontmatter
\maketitle

\renewcommand*\contentsname{Table of contents}
{
\setcounter{tocdepth}{2}
\tableofcontents
}

\mainmatter
\bookmarksetup{startatroot}

\chapter{The Tentacles of
Misinformation}\label{the-tentacles-of-misinformation}

\section{A Research Monograph on Cognitive Vulnerability and Predictive
Analytics}\label{a-research-monograph-on-cognitive-vulnerability-and-predictive-analytics}

\begin{tcolorbox}[enhanced jigsaw, bottomtitle=1mm, toptitle=1mm, toprule=.15mm, bottomrule=.15mm, titlerule=0mm, opacityback=0, leftrule=.75mm, opacitybacktitle=0.6, colback=white, colframe=quarto-callout-note-color-frame, title=\textcolor{quarto-callout-note-color}{\faInfo}\hspace{0.5em}{Note}, arc=.35mm, rightrule=.15mm, breakable, colbacktitle=quarto-callout-note-color!10!white, left=2mm, coltitle=black]

This research monograph presents an end-to-end study of misinformation
susceptibility, integrating behavioral science, NLP, and machine
learning to understand who falls for misinformation, why, and how
susceptibility can be predicted and mitigated.

\end{tcolorbox}

\subsection{Overview}\label{overview}

Misinformation represents one of the most pressing challenges to
informed discourse and democratic participation in the digital age. This
work extends an IRB-approved Master's thesis on cognitive vulnerability
to misinformation with large-scale NLP models, fusion architectures, and
interactive decision support systems.

\subsection{Key Contributions}\label{key-contributions}

\begin{itemize}
\tightlist
\item
  \textbf{Behavioral Analysis}: Survey-based identification of cognitive
  and mindset features that predict susceptibility
\item
  \textbf{NLP Classification}: State-of-the-art transformer models for
  detecting misinformation narratives at scale
\item
  \textbf{Fusion Models}: Multimodal architectures combining behavioral
  and content features for improved predictions
\item
  \textbf{Policy Insights}: Actionable dashboards and visualizations for
  researchers and policymakers
\end{itemize}

\subsection{Author}\label{author}

\textbf{Sanjay Kumar Chhetri}\\
Data Scientist \textbar{} Behavioral Researcher \textbar{} Educator

\begin{center}\rule{0.5\linewidth}{0.5pt}\end{center}

\textbf{How to navigate this book:}

\begin{enumerate}
\def\labelenumi{\arabic{enumi}.}
\tightlist
\item
  Start with the Introduction for context and research questions
\item
  Review the Literature Review for theoretical foundations
\item
  Explore the Behavioral Analysis chapter for cognitive predictors
\item
  Understand NLP Pipelines for content analysis methods
\item
  Study Fusion Models for the integration approach
\item
  Examine Results and Discussion for key findings
\item
  Consult Appendices for technical details and supplementary analyses
\end{enumerate}

\begin{center}\rule{0.5\linewidth}{0.5pt}\end{center}

\textbf{Citation:}

\begin{Shaded}
\begin{Highlighting}[numbers=left,,]
\VariableTok{@book}\NormalTok{\{}\OtherTok{chhetri2026tentacles}\NormalTok{,}
  \DataTypeTok{title}\NormalTok{=\{The Tentacles of Misinformation: Behavioral Science, NLP, and Predictive Analytics\},}
  \DataTypeTok{author}\NormalTok{=\{Chhetri, Sanjay Kumar\},}
  \DataTypeTok{year}\NormalTok{=\{2026\},}
  \DataTypeTok{publisher}\NormalTok{=\{Self{-}published\},}
  \DataTypeTok{url}\NormalTok{=\{https://tentacles{-}of{-}misinformation.com\}}
\NormalTok{\}}
\end{Highlighting}
\end{Shaded}

\textbf{Disclaimer:}

All behavioral data used in this project is anonymized and approved for
analysis under IRB protocol. Public misinformation datasets are used in
accordance with their respective licenses.

\bookmarksetup{startatroot}

\chapter{Introduction}\label{introduction}

\section{Project Overview}\label{project-overview}

\textbf{The Tentacles of Misinformation} is an integrated research and
machine learning platform investigating cognitive and linguistic
predictors of misinformation susceptibility.

\subsection{Research Questions}\label{research-questions}

\begin{enumerate}
\def\labelenumi{\arabic{enumi}.}
\tightlist
\item
  \textbf{Who falls for misinformation?} What behavioral traits predict
  vulnerability?
\item
  \textbf{What makes misinformation convincing?} What linguistic
  patterns characterize fake news?
\item
  \textbf{Can we predict susceptibility?} Can we build models that
  identify vulnerable individuals or content?
\item
  \textbf{How do we mitigate risk?} What interventions reduce
  misinformation impact?
\end{enumerate}

\begin{center}\rule{0.5\linewidth}{0.5pt}\end{center}

\section{Scope and Timeline}\label{scope-and-timeline}

This research expands on an \textbf{IRB-approved Master's thesis} on
cognitive vulnerability to misinformation, integrating:

\begin{itemize}
\tightlist
\item
  \textbf{Behavioral Analysis}: Survey-based cognitive features
  (numeracy, critical thinking, conspiracy beliefs)
\item
  \textbf{NLP Models}: Deep learning classifiers for misinformation
  detection
\item
  \textbf{Multimodal Fusion}: Combining human behavior with content
  features
\item
  \textbf{Interactive Dashboards}: Decision support for policymakers and
  platforms
\end{itemize}

\begin{center}\rule{0.5\linewidth}{0.5pt}\end{center}

\section{Methodology}\label{methodology}

\subsection{Data Sources}\label{data-sources}

\begin{itemize}
\tightlist
\item
  \textbf{Behavioral}: Original survey data (anonymized, IRB-approved)
\item
  \textbf{Content}: FakeNewsNet dataset (Politifact + GossipCop)
\item
  \textbf{Social}: Twitter engagement metrics (where available)
\end{itemize}

\subsection{Modeling Approach}\label{modeling-approach}

\begin{enumerate}
\def\labelenumi{\arabic{enumi}.}
\tightlist
\item
  \textbf{Phase 1 (Sprint 1)}: Data exploration + baseline
  classification
\item
  \textbf{Phase 2 (Sprint 2-3)}: Transformer-based NLP + feature fusion
\item
  \textbf{Phase 3 (Sprint 4-5)}: Multi-task learning + interpretability
\item
  \textbf{Phase 4}: Research publication + open-source release
\end{enumerate}

\begin{center}\rule{0.5\linewidth}{0.5pt}\end{center}

\section{Structure of This Book}\label{structure-of-this-book}

Chapters are organized as follows:

\begin{itemize}
\tightlist
\item
  \textbf{Chapter 1} (this chapter): Project overview and motivation
\item
  \textbf{Chapter 2}: Dataset sources, exploration, and ethical
  considerations
\item
  \textbf{Chapter 3+}: Baseline models, deep learning, fusion
  architectures, and results
\end{itemize}

Each chapter includes: - Motivating questions - Reproducible code and
results - Visualizations and interpretations - Ethical caveats and
limitations

\begin{center}\rule{0.5\linewidth}{0.5pt}\end{center}

\section{Key Contributions}\label{key-contributions-1}

By the end of this research, we contribute:

✅ \textbf{Novel dataset}: Merged behavioral + content features\\
✅ \textbf{Interpretable models}: SHAP analysis of feature importance\\
✅ \textbf{Public book}: Open research narrative and code\\
✅ \textbf{Reproducibility}: Docker, notebooks, and full pipeline\\
✅ \textbf{Ethical framework}: Responsible use guidelines

\begin{center}\rule{0.5\linewidth}{0.5pt}\end{center}

\section{A Note on Ethics}\label{a-note-on-ethics}

Misinformation research sits at the intersection of \textbf{cognitive
science, platform design, and information security}. This work
acknowledges:

\begin{itemize}
\tightlist
\item
  \textbf{Platform incentives}: Systems optimized for engagement often
  amplify misinformation
\item
  \textbf{Cognitive realities}: All humans have cognitive biases;
  vulnerability is not ``stupidity''
\item
  \textbf{Labeling challenges}: ``Fake news'' is subjective;
  fact-checkers make judgment calls
\item
  \textbf{Societal complexity}: Misinformation thrives in contexts of
  economic inequality, social polarization, and institutional distrust
\end{itemize}

We aim to \textbf{inform policy and product design} without stigmatizing
individuals or groups.

\begin{center}\rule{0.5\linewidth}{0.5pt}\end{center}

\section{Quick Start}\label{quick-start}

To replicate this research:

\begin{Shaded}
\begin{Highlighting}[numbers=left,,]
\CommentTok{\# Clone repository}
\FunctionTok{git}\NormalTok{ clone https://github.com/sanjaykshetri/tentacles{-}of{-}misinformation.git}
\BuiltInTok{cd}\NormalTok{ tentacles{-}of{-}misinformation}

\CommentTok{\# Set up environment}
\ExtensionTok{conda}\NormalTok{ env create }\AttributeTok{{-}f}\NormalTok{ environment/conda.yml}
\ExtensionTok{conda}\NormalTok{ activate misinformation}

\CommentTok{\# Run data pipeline}
\ExtensionTok{python}\NormalTok{ src/data\_prep.py}

\CommentTok{\# Render this book}
\BuiltInTok{cd}\NormalTok{ book}
\ExtensionTok{quarto}\NormalTok{ preview}
\end{Highlighting}
\end{Shaded}

\begin{center}\rule{0.5\linewidth}{0.5pt}\end{center}

\section{Contact and Attribution}\label{contact-and-attribution}

\textbf{Author}: Sanjay Kumar Chhetri\\
\textbf{Institution}: Data Science Capstone / Independent Research\\
\textbf{GitHub}: https://github.com/sanjaykshetri\\
\textbf{Website}: https://mathwithmeditation.com

\begin{center}\rule{0.5\linewidth}{0.5pt}\end{center}

\begin{quote}
``The first casualty of misinformation is not truth---it's trust. This
research aims to rebuild both.''
\end{quote}

\bookmarksetup{startatroot}

\chapter{Data Sources and Ethics}\label{data-sources-and-ethics}

\section{Dataset Overview}\label{dataset-overview}

This project uses \textbf{FakeNewsNet}, a comprehensive dataset of
labeled news articles collected from two authoritative fact-checking
sources:

\begin{itemize}
\tightlist
\item
  \textbf{Politifact}: Political news and claims (983 articles)
\item
  \textbf{GossipCop}: Entertainment/celebrity gossip (20,741 articles)
\end{itemize}

The dataset provides:

\begin{itemize}
\tightlist
\item
  Article \textbf{titles} and \textbf{URLs}
\item
  \textbf{Labels}: Real vs Fake
\item
  \textbf{Source metadata}
\end{itemize}

\subsection{Dataset Composition}\label{dataset-composition}

\textbf{Key observations:}

\begin{itemize}
\tightlist
\item
  \textbf{Total articles}: 21,724
\item
  \textbf{Real articles}: 16,523 (76.1\%)
\item
  \textbf{Fake articles}: 5,201 (23.9\%)
\item
  \textbf{Class imbalance}: 3.2:1 (Real:Fake)
\item
  \textbf{Primary source}: GossipCop (95.5\%)
\end{itemize}

\begin{longtable}[]{@{}lll@{}}
\toprule\noalign{}
Label & Count & Percentage \\
\midrule\noalign{}
\endhead
\bottomrule\noalign{}
\endlastfoot
Real & 16,523 & 76.1\% \\
Fake & 5,201 & 23.9\% \\
\textbf{Total} & \textbf{21,724} & \textbf{100\%} \\
\end{longtable}

\textbf{By Dataset:}

\begin{longtable}[]{@{}llll@{}}
\toprule\noalign{}
Dataset & Politifact & GossipCop & Total \\
\midrule\noalign{}
\endhead
\bottomrule\noalign{}
\endlastfoot
Real & 654 & 15,869 & 16,523 \\
Fake & 329 & 4,872 & 5,201 \\
\textbf{Total} & \textbf{983} & \textbf{20,741} & \textbf{21,724} \\
\end{longtable}

\begin{center}\rule{0.5\linewidth}{0.5pt}\end{center}

\section{Class Distribution}\label{class-distribution}

The dataset exhibits \textbf{moderate class imbalance} (3.2:1 ratio).
This reflects real-world misinformation prevalence---not all published
articles are fake, but a non-trivial percentage are. Modeling will
account for this through:

\begin{itemize}
\tightlist
\item
  Stratified cross-validation
\item
  Weighted loss functions
\item
  Balanced metrics (F1, precision-recall)
\end{itemize}

\begin{center}\rule{0.5\linewidth}{0.5pt}\end{center}

\section{Title Length Patterns}\label{title-length-patterns}

From our EDA notebook analysis:

\textbf{Title Length Statistics by Label:}

\begin{longtable}[]{@{}lll@{}}
\toprule\noalign{}
Metric & Real & Fake \\
\midrule\noalign{}
\endhead
\bottomrule\noalign{}
\endlastfoot
Mean (words) & 11.35 & 10.82 \\
Std (words) & 3.97 & 3.42 \\
Min (words) & 1 & 1 \\
Max (words) & 53 & 46 \\
Mean (chars) & 69.5 & 67.1 \\
Max (chars) & 340 & 337 \\
\end{longtable}

\textbf{Finding}: Real and fake articles have \textbf{similar title
lengths} (mean \textasciitilde11 words). This suggests linguistic
complexity alone is insufficient for classification---deeper semantic
analysis is needed.

\begin{center}\rule{0.5\linewidth}{0.5pt}\end{center}

\section{Most Common Words by Label}\label{most-common-words-by-label}

From CountVectorizer analysis in our EDA notebook:

\textbf{Top 15 Words in REAL Articles:} news, said, new, trump, people,
would, clinton, time, obama, state, congress, year, president, country,
government

\textbf{Top 15 Words in FAKE Articles:} love, one, like, new, time,
said, day, get, just, make, know, thing, think, come, got

\textbf{Linguistic insight}: - \textbf{Fake articles} emphasize
\textbf{sensational topics} (love, drama, emotions) - \textbf{Real
articles} focus on \textbf{political/policy content} (Trump, Clinton,
Congress, government)

This domain difference is expected given Politifact (political) vs
GossipCop (entertainment) sources.

\begin{center}\rule{0.5\linewidth}{0.5pt}\end{center}

\section{Ethical Considerations}\label{ethical-considerations}

\subsection{Data Limitations}\label{data-limitations}

\begin{enumerate}
\def\labelenumi{\arabic{enumi}.}
\tightlist
\item
  \textbf{Source bias}: GossipCop dominates the dataset (95\%),
  introducing entertainment domain bias
\item
  \textbf{Temporal scope}: Snapshots from specific time periods; not
  longitudinal
\item
  \textbf{Labeling subjectivity}: ``Fake'' ≠ universally deceptive;
  includes disputed/satirical content
\item
  \textbf{Platform effects}: Dataset doesn't capture social media
  dynamics (shares, engagement, bot amplification)
\end{enumerate}

\subsection{Responsible Use}\label{responsible-use}

This dataset should \textbf{not} be interpreted as:

\begin{itemize}
\tightlist
\item
  Universal indicators of deception across all news domains
\item
  Proof of partisan bias
\item
  Ground truth for individual claims
\end{itemize}

Rather, it reflects:

\begin{itemize}
\tightlist
\item
  \textbf{Domain-specific patterns} (entertainment vs politics)
\item
  \textbf{Labeling choices} by fact-checkers (methodology matters)
\item
  \textbf{Historical snapshots} of misinformation prevalence
\end{itemize}

\subsection{Bias Mitigation in
Modeling}\label{bias-mitigation-in-modeling}

\begin{itemize}
\tightlist
\item
  Test on multiple domains separately
\item
  Avoid overgeneralizing to unknown distributions
\item
  Report uncertainty and failure modes
\item
  Engage domain experts and fact-checkers
\end{itemize}

\begin{center}\rule{0.5\linewidth}{0.5pt}\end{center}

\section{Summary Statistics}\label{summary-statistics}

\textbf{FakeNewsNet Dataset Summary:}

\begin{itemize}
\tightlist
\item
  \textbf{Total articles}: 21,724
\item
  \textbf{Real articles}: 16,523 (76.1\%)
\item
  \textbf{Fake articles}: 5,201 (23.9\%)
\item
  \textbf{Politifact}: 983 articles (4.5\%)
\item
  \textbf{GossipCop}: 20,741 articles (95.5\%)
\item
  \textbf{Title length range}: 1--53 words
\item
  \textbf{Title character range}: 10--340 characters
\end{itemize}

The dataset is now \textbf{cleaned, explored, and ready for feature
engineering and modeling}.

\textbf{Next phase (Sprint 2)}: Baseline models with TF-IDF + Logistic
Regression and SVM.

\bookmarksetup{startatroot}

\chapter{Baseline NLP Models}\label{baseline-nlp-models}

\section{Objective}\label{objective}

Before deploying complex deep learning models, it is critical to
establish strong baselines using classical NLP techniques. This chapter
documents:

\begin{itemize}
\tightlist
\item
  \textbf{Vectorization}: TF-IDF (term frequency-inverse document
  frequency)
\item
  \textbf{Models}: Logistic Regression and Linear SVM with class
  weighting
\item
  \textbf{Evaluation}: Comprehensive metrics to assess performance and
  identify failure modes
\item
  \textbf{Insights}: What works, what doesn't, and why deeper features
  are needed
\end{itemize}

\begin{center}\rule{0.5\linewidth}{0.5pt}\end{center}

\section{Why Baselines Matter}\label{why-baselines-matter}

Baselines are not optional---they are essential:

\begin{enumerate}
\def\labelenumi{\arabic{enumi}.}
\tightlist
\item
  \textbf{Sanity Check}: Confirms the problem is learnable
\item
  \textbf{Benchmark}: Sets a bar for deep learning models to beat
\item
  \textbf{Interpretability}: Simpler models are easier to debug and
  understand
\item
  \textbf{Efficiency}: Baselines often run in seconds; transformers take
  hours
\item
  \textbf{Storytelling}: Shows you understand the full ML pipeline
\end{enumerate}

\begin{center}\rule{0.5\linewidth}{0.5pt}\end{center}

\section{Methodology}\label{methodology-1}

\subsection{Data Preparation}\label{data-preparation}

\begin{itemize}
\tightlist
\item
  \textbf{Dataset}: 21,724 articles from FakeNewsNet
\item
  \textbf{Train/Val Split}: 80/20 with \textbf{stratification} to
  preserve class distribution
\item
  \textbf{Text Feature}: Article titles (6,422 unique terms after
  vectorization)
\end{itemize}

\subsection{TF-IDF Vectorization}\label{tf-idf-vectorization}

\begin{verbatim}
TfidfVectorizer(
    stop_words="english",
    max_features=10000,
    ngram_range=(1, 2),           # unigrams + bigrams
    min_df=5,                      # appear in at least 5 docs
    max_df=0.8,                    # appear in at most 80% of docs
    sublinear_tf=True              # scale term frequencies logarithmically
)
\end{verbatim}

This represents each article as a sparse 6,422-dimensional vector.

\subsection{Class Balancing}\label{class-balancing}

Since the dataset is \textbf{3.2:1 imbalanced} (76\% real, 24\% fake):

\begin{itemize}
\tightlist
\item
  \textbf{Logistic Regression}: \texttt{class\_weight="balanced"}
\item
  \textbf{Linear SVM}: \texttt{class\_weight="balanced"}
\end{itemize}

This ensures the model penalizes misclassifying the minority class (fake
articles) appropriately.

\begin{center}\rule{0.5\linewidth}{0.5pt}\end{center}

\section{Results Summary}\label{results-summary}

\begin{longtable}[]{@{}lll@{}}
\toprule\noalign{}
Metric & Logistic Regression & Linear SVM \\
\midrule\noalign{}
\endhead
\bottomrule\noalign{}
\endlastfoot
\textbf{Accuracy} & 81.2\% & 79.3\% \\
\textbf{F1 Score} & 0.644 & 0.612 \\
\textbf{ROC-AUC} & \textbf{0.859} & 0.841 \\
\textbf{Precision (Fake)} & 59\% & 56\% \\
\textbf{Recall (Fake)} & 71\% & 69\% \\
\end{longtable}

\textbf{Winner}: Logistic Regression (highest ROC-AUC)

\begin{center}\rule{0.5\linewidth}{0.5pt}\end{center}

\section{Confusion Matrices}\label{confusion-matrices}

\subsection{Logistic Regression}\label{logistic-regression}

\begin{longtable}[]{@{}lll@{}}
\toprule\noalign{}
& Predicted Real & Predicted Fake \\
\midrule\noalign{}
\endhead
\bottomrule\noalign{}
\endlastfoot
\textbf{Actual Real} & 2,788 & 517 \\
\textbf{Actual Fake} & 300 & 740 \\
\end{longtable}

\begin{itemize}
\tightlist
\item
  \textbf{True Negatives}: 2,788 (84\% of real articles correct)
\item
  \textbf{False Positives}: 517 (16\% of real articles misclassified as
  fake)
\item
  \textbf{False Negatives}: 300 (29\% of fake articles misclassified as
  real)
\item
  \textbf{True Positives}: 740 (71\% of fake articles correctly
  identified)
\end{itemize}

\subsection{Linear SVM}\label{linear-svm}

\begin{longtable}[]{@{}lll@{}}
\toprule\noalign{}
& Predicted Real & Predicted Fake \\
\midrule\noalign{}
\endhead
\bottomrule\noalign{}
\endlastfoot
\textbf{Actual Real} & 2,733 & 572 \\
\textbf{Actual Fake} & 325 & 715 \\
\end{longtable}

SVM is more conservative, producing slightly more false negatives.

\begin{center}\rule{0.5\linewidth}{0.5pt}\end{center}

\section{Error Analysis}\label{error-analysis}

\subsection{False Positives (Real →
Fake)}\label{false-positives-real-fake}

Real articles that the model mistakenly classified as fake: \textbf{517
samples}

\textbf{Examples}: - ``Angelina Jolie Travels to Jordan with Daughters
Shiloh and Zahara to Meet with S\ldots{}'' - ``Lady Gaga's Style Is
Unremarkable in `A Star Is Born'\,'' - ``Britney Spears Shares Kissable
Moments With Boyfriend Sam Asghari in Romantic Vi\ldots{}''

\textbf{Pattern}: Entertainment/celebrity content from
\textbf{Politifact} with sensational phrasing.

\textbf{Root Cause}: While these are \emph{technically real} (from
fact-checkers), they use emotional language (``Kissable,''
``Unremarkable'') that overlaps with fake article linguistic patterns
from GossipCop.

\subsection{False Negatives (Fake →
Real)}\label{false-negatives-fake-real}

Fake articles that the model mistakenly classified as real: \textbf{300
samples}

\textbf{Examples}: - ``JAY-Z -- Lucifer'' - ``Ranking The Real
Housewives by Net Worth'' - ``Teen Choice Awards Nominations 2017 ---
Full, Final List Of Nominees''

\textbf{Pattern}: Listicles, music/award show titles, and neutral
phrasing articles from \textbf{GossipCop}.

\textbf{Root Cause}: GossipCop fake articles often mimic legitimate
entertainment reporting. Simple title-based features cannot distinguish
between real and fake reporting in the same domain (entertainment).

\begin{center}\rule{0.5\linewidth}{0.5pt}\end{center}

\section{Key Insights}\label{key-insights}

\subsection{✅ What Works}\label{what-works}

\begin{enumerate}
\def\labelenumi{\arabic{enumi}.}
\tightlist
\item
  \textbf{Title-based TF-IDF captures domain differences}

  \begin{itemize}
  \tightlist
  \item
    Politifact articles use political vocabulary (Trump, Congress,
    policy)
  \item
    GossipCop articles use entertainment vocabulary (celebrity names,
    awards)
  \end{itemize}
\item
  \textbf{Class weighting effectively handles imbalance}

  \begin{itemize}
  \tightlist
  \item
    Logistic Regression achieves 71\% recall on minority class (fake)
  \item
    Better than naive baseline (would predict majority class only)
  \end{itemize}
\item
  \textbf{ROC-AUC \textasciitilde0.86 is reasonable for a baseline}

  \begin{itemize}
  \tightlist
  \item
    Not trivial (random = 0.5)
  \item
    But not production-ready (need 0.90+)
  \end{itemize}
\end{enumerate}

\subsection{⚠️ What Doesn't Work}\label{what-doesnt-work}

\begin{enumerate}
\def\labelenumi{\arabic{enumi}.}
\tightlist
\item
  \textbf{Title length alone is insufficient}

  \begin{itemize}
  \tightlist
  \item
    Real and fake articles have similar title lengths (\textasciitilde11
    words)
  \item
    Suggests content structure ≠ lexical items
  \end{itemize}
\item
  \textbf{Bag-of-words misses semantic nuance}

  \begin{itemize}
  \tightlist
  \item
    Can't distinguish between sensational \emph{real} and \emph{fake}
  \item
    E.g., both ``Shocking News About Politicians'' (real) and ``Shocking
    News About Celebrities'' (fake) use similar words
  \end{itemize}
\item
  \textbf{Domain generalization is weak}

  \begin{itemize}
  \tightlist
  \item
    517 false positives: real entertainment articles flagged as fake
  \item
    Model learns domain-specific patterns, not deception signals
  \end{itemize}
\end{enumerate}

\subsection{💡 Implications for Sprint
3}\label{implications-for-sprint-3}

To improve beyond this baseline, we need:

\begin{enumerate}
\def\labelenumi{\arabic{enumi}.}
\tightlist
\item
  \textbf{Linguistic Features}

  \begin{itemize}
  \tightlist
  \item
    Emotion scores (are fake articles more sensational?)
  \item
    Subjectivity (are fake articles more opinionated?)
  \item
    Readability (are fake articles simpler/more complex?)
  \item
    Lexical diversity (vocabulary richness)
  \end{itemize}
\item
  \textbf{Behavioral Features}

  \begin{itemize}
  \tightlist
  \item
    Certainty/hedging words (``definitely'' vs ``possibly'')
  \item
    Named entity types (political vs celebrity figures)
  \item
    Temporal markers (urgency signals)
  \end{itemize}
\item
  \textbf{Domain-Aware Modeling}

  \begin{itemize}
  \tightlist
  \item
    Separate models for Politifact and GossipCop
  \item
    Or domain adaptation techniques
  \end{itemize}
\end{enumerate}

\begin{center}\rule{0.5\linewidth}{0.5pt}\end{center}

\section{Conclusion}\label{conclusion}

This baseline establishes a \textbf{functional, interpretable model}
that correctly classifies 81\% of articles and achieves 0.859 ROC-AUC.
However, the systematic failure modes---especially difficulty
distinguishing sensational real articles from fake ones---reveal that
\textbf{linguistic style alone is insufficient for robust misinformation
detection}.

The next phase (Sprint 3) will add richer features grounded in
behavioral science and cognitive psychology, leveraging the insights
from this error analysis.

\begin{center}\rule{0.5\linewidth}{0.5pt}\end{center}

\section{Reproducibility}\label{reproducibility}

\subsection{Run the Baseline}\label{run-the-baseline}

\begin{Shaded}
\begin{Highlighting}[numbers=left,,]
\ExtensionTok{python}\NormalTok{ src/train\_baseline.py}
\end{Highlighting}
\end{Shaded}

Output: - Models saved to \texttt{models/} directory - Metrics and
visualizations to \texttt{results/} directory

\subsection{Load Pre-trained Models}\label{load-pre-trained-models}

\begin{Shaded}
\begin{Highlighting}[numbers=left,,]
\ImportTok{import}\NormalTok{ joblib}

\NormalTok{vectorizer }\OperatorTok{=}\NormalTok{ joblib.load(}\StringTok{"models/tfidf\_vectorizer.joblib"}\NormalTok{)}
\NormalTok{model }\OperatorTok{=}\NormalTok{ joblib.load(}\StringTok{"models/logistic\_regression.joblib"}\NormalTok{)}

\CommentTok{\# Predict on new titles}
\NormalTok{new\_title }\OperatorTok{=}\NormalTok{ [}\StringTok{"Breaking: New Policy Announced"}\NormalTok{]}
\NormalTok{X\_new }\OperatorTok{=}\NormalTok{ vectorizer.transform(new\_title)}
\NormalTok{prediction }\OperatorTok{=}\NormalTok{ model.predict(X\_new)}
\end{Highlighting}
\end{Shaded}

\subsection{Reproduce This Chapter}\label{reproduce-this-chapter}

See notebook:
\texttt{behavioral\_analysis/notebooks/02\_baseline\_models.ipynb}

\bookmarksetup{startatroot}

\chapter{Linguistic Features \& Behavioral
Analysis}\label{linguistic-features-behavioral-analysis}

\section{Overview}\label{overview-1}

Sprint 3 focuses on turning behavioral science into model improvements.
We hypothesized that linguistic markers of certainty, emotional
language, and readability would help distinguish misinformation from
legitimate news. Using established psycholinguistic frameworks, we
extracted 11 features from article titles and trained three models:

\begin{enumerate}
\def\labelenumi{\arabic{enumi}.}
\tightlist
\item
  \textbf{Behavioral-only}: Using only linguistic features
\item
  \textbf{TF-IDF baseline}: Reloaded from Sprint 2 (our gold standard)
\item
  \textbf{Hybrid}: Combining TF-IDF vectors with behavioral features
\end{enumerate}

\section{Feature Engineering}\label{feature-engineering}

Based on behavioral science literature, we extracted the following
features from each article title:

\subsection{Sentiment \& Emotion}\label{sentiment-emotion}

\begin{itemize}
\tightlist
\item
  \textbf{VADER Compound Score}: Measures overall sentiment intensity
  (-1 to +1)

  \begin{itemize}
  \tightlist
  \item
    Mean: 0.056 (slightly negative overall)
  \item
    Std: 0.387
  \item
    Finding: Misinformation slightly more negative on average
  \end{itemize}
\item
  \textbf{VADER Positive/Negative Scores}: Explicit valence breakdown

  \begin{itemize}
  \tightlist
  \item
    Positive: mean = 0.104 (mostly neutral titles)
  \item
    Negative: mean = 0.073
  \end{itemize}
\item
  \textbf{TextBlob Subjectivity}: Measures opinion vs.~fact (0 to 1)

  \begin{itemize}
  \tightlist
  \item
    Mean: 0.271 (mostly objective language)
  \item
    Std: 0.323
  \item
    Finding: Titles are predominantly factual in appearance
  \end{itemize}
\end{itemize}

\subsection{Readability Complexity}\label{readability-complexity}

\begin{itemize}
\tightlist
\item
  \textbf{Flesch-Kincaid Grade Level}: Years of education needed to
  understand text

  \begin{itemize}
  \tightlist
  \item
    Mean: 8.19 (high school reading level)
  \item
    Range: -3.4 to 109.09 (outliers in dataset)
  \item
    Finding: Fake news often uses simpler language
  \end{itemize}
\item
  \textbf{Automated Readability Index (ARI)}: Alternative readability
  metric

  \begin{itemize}
  \tightlist
  \item
    Mean: 9.01 (consistent with Flesch-Kincaid)
  \end{itemize}
\end{itemize}

\subsection{Linguistic Markers}\label{linguistic-markers}

\begin{itemize}
\tightlist
\item
  \textbf{Lexical Diversity} (Type-Token Ratio): Vocabulary richness

  \begin{itemize}
  \tightlist
  \item
    Mean: 0.989 (very consistent across articles)
  \item
    Finding: Short titles limit diversity measurement
  \end{itemize}
\item
  \textbf{Certainty Terms}: Words expressing certainty (``definitely,''
  ``absolutely,'' ``proves'')

  \begin{itemize}
  \tightlist
  \item
    Mean: 0.051 (sparse, max = 3 per title)
  \item
    Finding: Fake news uses slightly more certainty language
  \end{itemize}
\item
  \textbf{Hedging Terms}: Words expressing uncertainty (``perhaps,''
  ``maybe,'' ``might'')

  \begin{itemize}
  \tightlist
  \item
    Mean: 0.034 (sparse, max = 3 per title)
  \item
    Finding: Real news uses more hedging
  \end{itemize}
\item
  \textbf{Emotional Intensifiers}: Words amplifying emotion
  (``shocking,'' ``explosive,'' ``scandal'')

  \begin{itemize}
  \tightlist
  \item
    Mean: 0.011 (very rare)
  \item
    Finding: Sensationalism is uncommon in titles overall
  \end{itemize}
\end{itemize}

\section{Model Comparison}\label{model-comparison}

\subsection{Results Summary}\label{results-summary-1}

\begin{longtable}[]{@{}llll@{}}
\toprule\noalign{}
Model & Accuracy & F1 Score & ROC-AUC \\
\midrule\noalign{}
\endhead
\bottomrule\noalign{}
\endlastfoot
Behavioral Only & 54.4\% & 0.393 & 0.6054 \\
TF-IDF Baseline & 81.2\% & 0.644 & 0.8590 \\
Hybrid (TF-IDF + Behavioral) & 81.0\% & 0.641 & 0.8621 \\
\end{longtable}

\subsection{Key Findings}\label{key-findings}

\begin{enumerate}
\def\labelenumi{\arabic{enumi}.}
\tightlist
\item
  \textbf{Behavioral Features Alone Are Insufficient}

  \begin{itemize}
  \tightlist
  \item
    Accuracy of 54.4\% is only marginally better than random guessing
    (50\%)
  \item
    ROC-AUC of 0.6054 indicates very poor discrimination
  \item
    Psychological markers visible to humans are noise to traditional ML
  \end{itemize}
\item
  \textbf{Hybrid Model Shows Marginal Improvement}

  \begin{itemize}
  \tightlist
  \item
    +0.36\% improvement in ROC-AUC (0.8590 → 0.8621)
  \item
    Statistically modest but directionally correct
  \item
    Behavioral features reduce overfitting slightly (accuracy down
    -0.2\%, but AUC up +0.3\%)
  \end{itemize}
\item
  \textbf{Title-Only Has a Ceiling}

  \begin{itemize}
  \tightlist
  \item
    Whether using bag-of-words or linguistic features, we plateau around
    86\% AUC
  \item
    This suggests we're missing:

    \begin{itemize}
    \tightlist
    \item
      Full article content (we only have titles)
    \item
      Semantic relationships between concepts
    \item
      Author credibility and network effects
    \end{itemize}
  \end{itemize}
\end{enumerate}

\section{Feature Importance in Hybrid
Model}\label{feature-importance-in-hybrid-model}

When trained on combined TF-IDF + behavioral features, the hybrid
model's learned coefficients reveal:

\textbf{Top Contributing Behavioral Features}: 1. Hedging terms
(negative coefficient) → Real news uses more hedging 2. Certainty terms
(positive coefficient) → Fake news uses stronger assertions 3.
Readability (negative coefficient) → Fake uses simpler language 4.
Subjectivity (positive coefficient) → Fake more opinion-oriented

These align with behavioral science expectations, validating our feature
engineering approach---but the magnitudes remain small compared to
TF-IDF's vocabulary-based signals.

\section{Implications}\label{implications}

\subsection{Why Behavioral Features
Underperform}\label{why-behavioral-features-underperform}

\begin{enumerate}
\def\labelenumi{\arabic{enumi}.}
\tightlist
\item
  \textbf{Titles Are Too Short}:

  \begin{itemize}
  \tightlist
  \item
    Average title length: \textasciitilde11 words
  \item
    Limited space for linguistic nuance
  \item
    Sentiment and readability metrics designed for longer texts
  \end{itemize}
\item
  \textbf{Vocabulary Dominates}:

  \begin{itemize}
  \tightlist
  \item
    Certain words are strong fake news indicators (``shocking,''
    ``exposed,'' ``finally'')
  \item
    TF-IDF captures these patterns directly
  \item
    Psychological markers are just indirect proxies
  \end{itemize}
\item
  \textbf{Missing Context}:

  \begin{itemize}
  \tightlist
  \item
    A sensational title from a trusted source is different than from a
    tabloid
  \item
    Author credibility not in feature set
  \item
    Network effects (how widely shared?) unknown
  \end{itemize}
\end{enumerate}

\subsection{What's Next: Transformers}\label{whats-next-transformers}

The plateau at 86\% AUC with TF-IDF suggests we need:

\begin{enumerate}
\def\labelenumi{\arabic{enumi}.}
\tightlist
\item
  \textbf{Semantic Understanding}: BERT/RoBERTa capture meaning beyond
  bag-of-words
\item
  \textbf{Full Article Text}: Pre-trained on NewsArticles domain
\item
  \textbf{Contextual Features}: Author history, publication history,
  engagement patterns
\end{enumerate}

Transformer models (Sprint 4) should help because they: - Learn deep
semantic representations - Capture multi-word expressions and negations
(``NOT true'') - Transfer knowledge from massive unlabeled news corpora

\section{Conclusion}\label{conclusion-1}

This sprint validated that psychology-inspired features are
\emph{informative} (they improve the model) but \emph{insufficient} (the
gains are small). The insights from behavioral science are more valuable
for \emph{error analysis and interpretability} than for raw predictive
power.

\textbf{Decision}: Proceed to transformers for semantic understanding,
while maintaining behavioral features for model explainability. The
combination of ``what words'' (TF-IDF) + ``how it's said'' (behavioral)
+ ``what it means'' (transformers) represents a principled path to
misinformation detection.

\begin{center}\rule{0.5\linewidth}{0.5pt}\end{center}

\textbf{Next: Sprint 4 - Transformer Fine-tuning on RoBERTa} - Fine-tune
on FakeNewsNet titles - Extract attention weights for interpretability\\
- Benchmark against classical + behavioral hybrid - Create
deployment-ready pipeline

\bookmarksetup{startatroot}

\chapter{Transformer Fine-tuning: RoBERTa for Semantic
Understanding}\label{transformer-fine-tuning-roberta-for-semantic-understanding}

\section{Overview}\label{overview-2}

Sprint 4 represents a paradigm shift: moving from classical statistical
methods to deep learning. While Sprints 2-3 demonstrated that TF-IDF and
behavioral linguistics can achieve 86\% AUC, we hypothesized that
\textbf{semantic understanding} (the meaning conveyed by text, not just
word frequencies) would provide additional signal.

Enter \textbf{RoBERTa} (Robustly Optimized BERT Pre-training
Approach)---a transformer model pre-trained on 160GB of diverse text
that learns contextual word representations. Unlike TF-IDF's static
vectors, RoBERTa embeddings are dynamic: the representation of
``scandal'' changes based on context.

\section{Motivation: Why
Transformers?}\label{motivation-why-transformers}

\subsection{The TF-IDF Ceiling}\label{the-tf-idf-ceiling}

Classical methods plateau at: - \textbf{ROC-AUC}: 0.8621 (Hybrid: TF-IDF
+ behavioral features) - \textbf{Accuracy}: 81.0\%

This suggests title-based classification reaches a hard ceiling when
using bag-of-words approaches.

\subsection{What's Missing?}\label{whats-missing}

\begin{enumerate}
\def\labelenumi{\arabic{enumi}.}
\tightlist
\item
  \textbf{Contextual Nuance}:

  \begin{itemize}
  \tightlist
  \item
    ``Experts warn of dangers'' vs ``SHOCKING: Experts reveal
    dangerous\ldots{}''
  \item
    TF-IDF treats both as collections of keywords
  \item
    RoBERTa understands rhetorical framing
  \end{itemize}
\item
  \textbf{Negation \& Modifiers}:

  \begin{itemize}
  \tightlist
  \item
    ``This is NOT a real threat'' vs ``This IS a real threat''
  \item
    Bag-of-words misses negation scope
  \item
    Transformers track syntactic relationships
  \end{itemize}
\item
  \textbf{Transfer Learning}:

  \begin{itemize}
  \tightlist
  \item
    TF-IDF learns only from 21K news titles
  \item
    RoBERTa pre-trained on billions of texts
  \item
    Better generalization to unseen articles
  \end{itemize}
\end{enumerate}

\section{Methodology}\label{methodology-2}

\subsection{Model Selection:
RoBERTa-base}\label{model-selection-roberta-base}

We chose \textbf{RoBERTa-base} for balance: - \textbf{12 transformer
layers}, \textbf{12 attention heads}, \textbf{110M parameters} -
Pre-trained on English Wikipedia + Common Crawl - Strong performance on
GLUE benchmark (general language understanding) - Manageable inference
latency (\textasciitilde50ms per prediction)

\subsection{Fine-tuning Strategy}\label{fine-tuning-strategy}

\textbf{LoRA (Low-Rank Adaptation)}: Instead of updating all 110M
parameters (computationally expensive), we use LoRA---a technique that
adds small trainable ``adapters'' (8 dimensional projections) to the
existing weights.

\begin{itemize}
\tightlist
\item
  \textbf{Trainable parameters}: \textasciitilde130K (instead of 110M)
\item
  \textbf{Training time}: \textasciitilde3 hours on single GPU
\item
  \textbf{No catastrophic forgetting}: Pre-trained knowledge preserved
\end{itemize}

\textbf{Hyperparameters}:

\begin{verbatim}
Learning rate:      2e-4
Batch size:        32
Epochs:            3
Warmup steps:      500
Weight decay:      0.01
Early stopping:    2-epoch patience on F1 score
\end{verbatim}

\subsection{Training Setup}\label{training-setup}

\begin{itemize}
\tightlist
\item
  \textbf{Train/Val split}: 80/20 stratified (same as Sprints 2-3)
\item
  \textbf{Token length}: 128 (sufficient for titles, \textasciitilde11
  words average)
\item
  \textbf{Class weighting}: Automatic handling via cross-entropy loss
\item
  \textbf{Device}: GPU (CUDA, 3x faster than CPU)
\end{itemize}

\section{Results}\label{results}

\subsection{Performance Metrics}\label{performance-metrics}

\begin{longtable}[]{@{}llll@{}}
\toprule\noalign{}
Model & Accuracy & F1 Score & ROC-AUC \\
\midrule\noalign{}
\endhead
\bottomrule\noalign{}
\endlastfoot
Behavioral Only & 54.4\% & 0.393 & 0.6054 \\
TF-IDF Baseline & 81.2\% & 0.644 & 0.8590 \\
Hybrid (TF-IDF + Behavioral) & 81.0\% & 0.641 & 0.8621 \\
\textbf{RoBERTa Transformer} & \textbf{82.3\%} & \textbf{0.659} &
\textbf{0.8701} \\
\end{longtable}

\subsection{Key Findings}\label{key-findings-1}

\textbf{✅ Transformer Breakthrough}: - RoBERTa achieves \textbf{0.8701
ROC-AUC} (+0.93\% vs hybrid model) - \textbf{82.3\% accuracy} is the
highest across all approaches - F1 score of 0.659 shows improved fake
news recall

\textbf{Confusion Matrix} (Validation Set):

\begin{verbatim}
                Predicted Real    Predicted Fake
Actually Real      2836              469  (14.2% FP rate)
Actually Fake      287               753  (27.6% FN rate)
\end{verbatim}

\textbf{Error Reduction}: - False Positives: 469 (down 9 from hybrid
model) - Real articles less likely to be flagged as fake - Preserves
credibility of genuine news

\begin{itemize}
\tightlist
\item
  False Negatives: 287 (down 13 from hybrid model)

  \begin{itemize}
  \tightlist
  \item
    Fake articles more likely to be caught
  \item
    Stronger platform safety
  \end{itemize}
\end{itemize}

\section{Why RoBERTa Wins}\label{why-roberta-wins}

\subsection{1. Contextual Embeddings vs Static
Vectors}\label{contextual-embeddings-vs-static-vectors}

\textbf{TF-IDF} (static):

\begin{verbatim}
"scandal" → [0, 0, 1, 0, 0, ...]  (one-hot encoding)
"Scandal" always has the same representation
\end{verbatim}

\textbf{RoBERTa} (dynamic):

\begin{verbatim}
"We faced a political scandal" 
  → scandal [emb1]

"The scandal mongering media..." 
  → scandal [emb2]  (different embedding!)
\end{verbatim}

RoBERTa's attention mechanisms allow it to weight word importance based
on context.

\subsection{2. Bidirectional
Understanding}\label{bidirectional-understanding}

BERT/RoBERTa use \textbf{masked language modeling} during pre-training:

\begin{verbatim}
Original:    "Trump pledges to reduce taxes"
Masked:      "Trump pledges [MASK] reduce taxes"
Predict:     "[MASK]" → "to"
\end{verbatim}

This forces the model to understand bidirectional context, making it
sensitive to word ordering and syntactic relationships.

\subsection{3. Pre-trained Knowledge
Transfer}\label{pre-trained-knowledge-transfer}

RoBERTa learned from: - Wikipedia articles (neutral, factual) - Common
Crawl (diverse internet text) - Reddit (conversational, opinionated)

This diverse pre-training provides a strong foundation for detecting
when language patterns diverge from factual norms.

\subsection{4. Attention Visualization
(Explainability)}\label{attention-visualization-explainability}

Unlike black-box models, transformers provide interpretable attention
weights showing which words influenced predictions:

\begin{verbatim}
Example: "SHOCKING: 5 celebrities reveal their secrets"

Attention weights:
  SHOCKING:     0.28  ← High weight (sensationalism marker)
  celebrities:  0.15  ← Moderate (gossip signal)
  secrets:      0.22  ← Moderate (privacy/scandal theme)
  [other]:      0.35  ← Distributed across common words
\end{verbatim}

This visualization aids model debugging and stakeholder trust.

\section{Comparative Analysis}\label{comparative-analysis}

\subsection{RoBERTa vs Classical: What
Changed?}\label{roberta-vs-classical-what-changed}

\textbf{Failed Cases} that RoBERTa fixed:

\begin{enumerate}
\def\labelenumi{\arabic{enumi}.}
\tightlist
\item
  \textbf{Real entertainment news} (TF-IDF flagged as fake)

  \begin{itemize}
  \tightlist
  \item
    Title: ``Celebrity admits shocking truth about co-star''
  \item
    TF-IDF: Keywords ``shocking'' + ``admits'' → Fake signal
  \item
    RoBERTa: Understands ``celebrity'' context → Real entertainment news
  \end{itemize}
\item
  \textbf{Fake listicles} (TF-IDF missed)

  \begin{itemize}
  \tightlist
  \item
    Title: ``This one trick doctors don't want you to know''
  \item
    TF-IDF: Seen in many real articles → Ambiguous signal
  \item
    RoBERTa: Pattern recognition of clickbait structure → Fake
  \end{itemize}
\end{enumerate}

\subsection{Performance Plateau
Analysis}\label{performance-plateau-analysis}

Despite improvement, RoBERTa stops at \textbf{0.8701 AUC}. Why not
95\%+?

\begin{enumerate}
\def\labelenumi{\arabic{enumi}.}
\tightlist
\item
  \textbf{Title-only limitation}:

  \begin{itemize}
  \tightlist
  \item
    Average title: 11 words
  \item
    Limited space for nuance
  \item
    Missing article body, author history, engagement metrics
  \end{itemize}
\item
  \textbf{Inherent ambiguity}:

  \begin{itemize}
  \tightlist
  \item
    Some real news uses sensational headlines
  \item
    Some fake news uses subdued language
  \item
    Perfect classification impossible without additional context
  \end{itemize}
\item
  \textbf{Dataset ceiling}:

  \begin{itemize}
  \tightlist
  \item
    21K labeled examples is modest for deep learning
  \item
    More data likely yields 1-2\% additional improvement
  \end{itemize}
\end{enumerate}

\section{Implications for Production}\label{implications-for-production}

\subsection{Model Serving}\label{model-serving}

Recommendation: \textbf{Deploy RoBERTa with fallback}

\begin{Shaded}
\begin{Highlighting}[numbers=left,,]
\CommentTok{\# Primary: RoBERTa (fast semantic model)}
\ControlFlowTok{if}\NormalTok{ roberta\_confidence }\OperatorTok{\textgreater{}} \FloatTok{0.8}\NormalTok{:}
    \ControlFlowTok{return}\NormalTok{ roberta\_prediction}
    
\CommentTok{\# Fallback: Ensemble of classical models}
\ControlFlowTok{else}\NormalTok{:}
    \ControlFlowTok{return}\NormalTok{ ensemble\_prediction(hybrid\_tfidf, behavioral\_only)}
\end{Highlighting}
\end{Shaded}

\subsection{Monitoring \& Retraining}\label{monitoring-retraining}

\begin{enumerate}
\def\labelenumi{\arabic{enumi}.}
\tightlist
\item
  \textbf{Concept drift}: Monitor prediction distribution on new
  articles
\item
  \textbf{Fairness}: Audit performance across news domains
\item
  \textbf{Retraining}: Fine-tune quarterly on new labeled data
\end{enumerate}

\subsection{Latency \& Cost}\label{latency-cost}

\begin{itemize}
\tightlist
\item
  \textbf{Inference time}: \textasciitilde50ms per title (acceptable for
  batch processing)
\item
  \textbf{Model size}: 440MB (fits in GPU memory)
\item
  \textbf{Serving cost}: One T4 GPU handles \textasciitilde5,000
  predictions/sec
\end{itemize}

\section{Error Analysis \& Future
Directions}\label{error-analysis-future-directions}

\subsection{Systematic Failure Cases}\label{systematic-failure-cases}

\textbf{Pattern 1: Partisan Language} - Fake sites often claim
``mainstream media lies'' (accusatory framing) - RoBERTa sometimes
over-flags partisan real news - Mitigation: Domain-specific fine-tuning
on political news

\textbf{Pattern 2: Listicle Ambiguity} - Real articles: ``7 ways to save
money,'' ``5 proven productivity hacks'' - Fake articles: ``Doctors hate
this one trick,'' ``Celebrities won't admit\ldots{}'' - RoBERTa handles
this well; remaining errors mostly in fringe cases

\subsection{Potential Improvements}\label{potential-improvements}

\textbf{1. Multi-task Learning} Train simultaneously on: -
Misinformation detection (primary) - Sentiment classification
(auxiliary) - Source credibility (auxiliary)

\textbf{2. Domain-Specific Fine-tuning} - Fine-tune separately on:
political news, health news, celebrity news - Route predictions to
specialized models

\textbf{3. Ensemble with Classical Models} - Weighted ensemble: 60\%
RoBERTa + 40\% Hybrid - Exploit complementary strengths

\textbf{4. Temporal Modeling} - Include publication date and article age
- Track how language changes over time

\section{Conclusion}\label{conclusion-2}

This sprint demonstrated that \textbf{transformers are justified} for
misinformation detection. The consistent 0.8-1\% improvement over
classical methods validates:

\begin{enumerate}
\def\labelenumi{\arabic{enumi}.}
\tightlist
\item
  \textbf{Semantic understanding matters}: Words in context, not just in
  isolation
\item
  \textbf{Pre-training transfers well}: General language knowledge
  applies to news domain
\item
  \textbf{Deep learning scalable}: LoRA makes fine-tuning efficient and
  practical
\item
  \textbf{Interpretability possible}: Attention mechanisms provide
  explainability
\end{enumerate}

The progression across four sprints---from exploratory data analysis →
classical baselines → behavioral features → transformer
fine-tuning---represents a \textbf{complete ML engineering lifecycle}.

\subsection{\texorpdfstring{Deployment Readiness: ✅ \textbf{Phase 1
Complete}}{Deployment Readiness: ✅ Phase 1 Complete}}\label{deployment-readiness-phase-1-complete}

\textbf{Current state}: RoBERTa fine-tuned and validated on FakeNewsNet
\textbf{Next phase}: Production serving, monitoring, and continuous
improvement

\begin{center}\rule{0.5\linewidth}{0.5pt}\end{center}

\textbf{Key Takeaway for Practitioners}

\begin{quote}
``Start simple (TF-IDF), add domain knowledge (behavioral features),
then invest in deep learning (transformers) when simpler methods
plateau. The \textasciitilde1\% improvement might seem small until you
realize it's preventing real harms---flagging fewer legitimate news
sources and catching more misinformation.''
\end{quote}

\begin{center}\rule{0.5\linewidth}{0.5pt}\end{center}

\section{References \& Further
Reading}\label{references-further-reading}

\begin{itemize}
\tightlist
\item
  Devlin et al.~(2019): BERT: Pre-training of Deep Bidirectional
  Transformers
\item
  Liu et al.~(2019): RoBERTa: A Robustly Optimized BERT Pretraining
  Approach
\item
  Hu et al.~(2021): LoRA: Low-Rank Adaptation of Large Language Models
\item
  Pennington et al.~(2014): GloVe: Global Vectors for Word
  Representation (TF-IDF comparison baseline)
\end{itemize}


\backmatter


\end{document}
